\documentclass[11pt]{article}
\usepackage{geometry} % see geometry.pdf on how to lay out the page. There's lots.
\usepackage{hyperref}
\usepackage{graphicx}
\usepackage{gensymb}
\usepackage[affil-it]{authblk}
\usepackage[toc,page]{appendix}
\usepackage{pifont}
\usepackage{amsmath}
\usepackage{amssymb}
\usepackage{relsize}
\usepackage{draftwatermark}

\SetWatermarkText{DRAFT}
\SetWatermarkScale{6}
\SetWatermarkLightness{0.95}

% \geometry{letter} % or letter or a5paper or ... etc
% \geometry{landscape} % rotated page geometry

% See the ``Article customise'' template for come common customisations

\title{On Simple Planar Variable Geometry Trusses}
\author{Robert L. Read
  \thanks{read.robert@gmail.com}
}
\affil{Founder, Public Invention, an educational non-profit.}


\date{\today}

%%% BEGIN DOCUMENT
\begin{document}

\maketitle

%% \tableofcontents

\section{Introduction}

A {\em variable geometry truss} is a truss in which changes shape by means of the change of the length of members, in contrast
to many robot arms which change joint angles. We call a member which can change length an {\em actuator}.
A truss constructed purely by starting with a triangle and and repeatedly adding two members and a new joint to one side of an existing triangle
to form a new triangle is a {\em simple truss}. This paper
is concerned only with simple planar trusses. In particular, we focus on trusses isomorphic to the Warren truss (e.g., an unbranching configuration of triangles.)
Furthermore, although the world {\em truss} connotes forces and structural analysis,
we are concerned purely with geometry. Altough motivated by robotics, we assume that our trusses and actuators are strong enough that
we need not consider the forces acting upon the truss---in other words, we are treating it as mechanism and not a machine.

If we imagine a joint, usually at one end of our truss, to be an end effector, 
he fundamental goal of this paper is to answer the question:
\begin{quote}
  How does a change in length of an actuator change the position of an end effector?
\end{quote}

\section{Formulation}

A {\em truss} structure is a graph and a set of fixed nodes $T=(G,F)$. $G = (V,E)$. $V$ is a set of nodes or joints. $E$ is a set of lines or members which are 2-element subsets of $V$.
At least two nodes are condidered to be fixed in space via a set of fixed nodes $F = {(i,x,y) \| i \in V \and x,y \in \mathbb{R}}$.

We number joints from zero and designate them with a subscript. Members are designated by two subscripts, naming the joints they connect.

A configuration function mapping each member of a truss structure to a non-negative length. $C: V \times V \mapsto \mathbb{R}$.

A geometry is a placement of each joint in the Cartesian plane. $G: V \mapsto \mathbb{R} \times \mathbb{R} $.

A {\em simple truss} is a truss constructed from a triangle by adding a single joint and two members to a side repetitively.
A {\em Warren truss} is an unbranched simple truss, which is isomorphic to a truss that is a single chain of equilateral triangles.
Furthermore, we insist that each even node $n$ occurs as a anti-clockwise turn (in the anti-clockwise semiplane) from the vector $\overrightarrow{n-2,n-1}$, and
each odd node occurs as a clockwise turn (in the clockwise semiplane) from the previous to nodes.

For a Warren truss, there is a simple algorithm $W$ for computing a geometry from a configuration: $W : C \mapsto G$.

A goal joint is a joint $j$ such that $j \in V$ and a desired position given by a goal function $d : V \mapsto \mathbb{R}^2$. In robotics,
a goal joint is often called an {\em end effector}.

A {\em scoring function} is a function that that takes a geometry and returns a real, non-negative value based on 
the goal node positions given by by a goal function and the actual position given by a Geometry. A score of zero is considered
 a perfect score and the a higher score represents a less sought-after result.

A {\em linear distance scoring function} is a special scoring function that that takes a geometry and returns a real, non-negative value based solely
on a summation of distances between the goal node positions given by by a goal function and the actual position given by a Geometry.

A truss {\em problem} $P = (T,d,s)$ is a truss with a scoring function and a desired position function.
A solution to a problem is a configuration and a geometry. An optimal solution is a solution which minimizes the score.

Our fundamental goal is to develop a formula for the partial derivative of the end effector
with respect to the change in length of a member.

A further goal is to be able to render a diagram illustrating the impact on the end-effector
of a change of each member, as drawn by rendering a vector from the center point of each member.

A further goal is to have algorithms to determine:
\begin{itemize}
\item What is the minimum overall change in length to a group of actuators to solve a Problem?
\item Can a Problem be solved with a single change in length?
\item Assuming bounds on the lengths of actuators, can we solve Problems?
  \item What is the workspace of a truss?
  \end{itemize}

In the remainder of this paper we will consider only Warren trusses, linear distance scoring functions, and desired position functions
that map only one node, called an end effector, to a desired position. Furthermore, we will assume that the first two nodes are fixed
and that the furthest node (by path length) from the first node is the end effector.

\section{Moving an External Member}

We seek a formula for the change in position of the end effector $e$ with respect to change in the length of a member given a geometry.
In the case of a Warren truss, all members can be divided between external members and internal members. A change to the length of an
external member is particularly simple.

The change in position is a centered on the goal node representing the place the goal node would move to with a unit change in length.
However, the derivative is only valid as an infinitessimal, but as an infinitessimal its magnitude and direction may be usefully
added to or compared to other such vectors. We can thus usefully tell which member would change the position of the goal node most
rapidly in response to a minute change in different members.

The fundamental observation is that for an external member $(i,i-2)$, a change in length generates a rotation $\theta$ about joint $n-1$
whose position is given by $(x,y)$. $\theta$ is the angle of the vector from the pivot point to the end-effector with the $x$-axis.
This rotation applies to the triangle defined by three joints: $\triangle i-1,i,e$. Not that this triangle doe not exist as a physical
structure in the truss. This rotation about a point can be expressed as a matrix $M$:

\[
M = 
\begin{bmatrix}
    \cos{\theta} & -\sin{\theta} & -x \cdot \cos{\theta} + x + y \cdot \sin{\theta}  \\
    \sin{\theta} & \cos{\theta} & -x \cdot \sin{\theta} - y \cdot \cos{\theta} + y  \\
    0 & 0 & 1 & 
\end{bmatrix}
\]

such that:
\[
M \begin{bmatrix}
           e_x \\
           e_y \\
           1
         \end{bmatrix} = \begin{bmatrix}
           e_x' \\
           e_y' \\
           1
         \end{bmatrix}
\]
where $(e_x',e_y')$ is the new position of $e$ after the rotation by $theta$. Multiplying this out:

\[
\begin{bmatrix}
           e_x' \\
           e_y' \\
         \end{bmatrix}
=
\begin{bmatrix}
           e_x \cos{\theta} + -e_y \sin{\theta}  + -x\cos{\theta} + x + y \sin{\theta} \\
           e_x \sin{\theta} + e_y \cos{\theta} + -x\sin{\theta} + - y \cos{\theta} + y \\
         \end{bmatrix}
\]
Collecting terms:
\[
\begin{bmatrix}
           e_x' \\
           e_y' \\
         \end{bmatrix}
=
\begin{bmatrix}
           (e_x  - x) \cos{\theta} + (y - e_y)\sin{\theta}  +  x  \\
           (e_x - x ) \sin{\theta} + (e_y - y)\cos{\theta} +  y \\
         \end{bmatrix}
\]
This can be easily differentiated with respect to $\theta$:


\[
\frac{\partial e}{\partial \theta} = \begin{bmatrix}
           \frac{\partial e_x}{\partial \theta} \\
           \frac{\partial e_y}{\partial \theta} \\
         \end{bmatrix} = \begin{bmatrix}
           -(e_x  - x) \sin{\theta} + -(e_y -y)\cos{\theta}  \\
           (e_x - x ) \cos{\theta} + -(e_y - y)\sin{\theta}  \\
         \end{bmatrix}
\]

As we might expect, the magnitude of this vector depends on the distance from the pivot joint $(x,y)$ and the position of the end effector
$(e_x,e_y)$ and the direction depends on the direction of the vector from the pivot joint to the end effector, $\theta$.

By using the law of cosines, where $a = \|\overrightarrow{n-2,n-1}\|, b = \|\overrightarrow{n-2,n}\| = l_{i,i-2}, c = \|\overrightarrow{n-1,n}\| $  ,
where $b$ is the member opposite the pivot joint $n-1$ which changes the angle $\angle n-2,n-1,n = \phi_{i-1}$.
\[
\cos{\phi_{i-1}} = \frac{a^2 - b^2 + c^2}{2 a c}
\]

Using Woflram Alpha to differentiate this, we obtain:

\[
\frac{\partial \phi_{i-1}}{\partial l_{i,i-2}} = \frac{b}{ac\sqrt{1 - \frac{(a^2 + c^2 - b^2)^2}{4a^2c^2}}}
\]

So, by the chain rule

\[
\frac{\partial e}{\partial  l_{i,i-2}} = \begin{bmatrix}
           \frac{\partial e_x}{\partial l_{i,i-2}} \\
           \frac{\partial e_y}{\partial l_{i,i-2}} \\
         \end{bmatrix} = \frac{b}{ac\sqrt{1 - \frac{(a^2 + c^2 - b^2)^2}{4a^2c^2}}} \begin{bmatrix}
           -(e_x  - x) \sin{\theta} + -(e_y -y)\cos{\theta}  \\
           (e_x - x ) \cos{\theta} + -(e_y - y)\sin{\theta}  \\
         \end{bmatrix}
\]

So this is a closed-form expression for the change in the position for any external member $i,i-2$ we choose.


\section{Moving an Internal Member}

Moving an internal member is slightly more complicated. In a Warren Truss, moving an internal member $(i,i-1)$ always
moves the large node around the the node $i-2$.

\section{References}



\end{document}

http://eprints.cs.vt.edu/archive/00000192/01/TR-90-10.pdf


https://people.eecs.berkeley.edu/~elghaoui/Teaching/EE227A/lecture6.pdf

http://homes.cs.washington.edu/~sagarwal/aat.pdf

http://zoonek.free.fr/blosxom/R/2012-06-01_Optimization.html

http://docs.mosek.com/modeling-cookbook/cqo.html

// This one looks pretty interesting
http://ieeexplore.ieee.org/abstract/document/4160811/

https://tspace.library.utoronto.ca/bitstream/1807/14046/1/NQ49815.pdf
